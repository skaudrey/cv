% resume.tex
% vim:set ft=tex spell:

\documentclass[10pt,letterpaper]{article}
\usepackage[letterpaper,margin=0.75in]{geometry}
\usepackage[utf8]{inputenc}
\usepackage{mdwlist}
\usepackage[T1]{fontenc}
\usepackage{textcomp}
\usepackage{tgpagella}
\usepackage{latexsym}
\usepackage{amssymb}
\usepackage{hyperref}
\pagestyle{empty}
\setlength{\tabcolsep}{0em}

% indentsection style, used for sections that aren't already in lists
% that need indentation to the level of all text in the document
\newenvironment{indentsection}[1]%
{\begin{list}{}%
	{\setlength{\leftmargin}{#1}}%
	\item[]%
}
{\end{list}}

% opposite of above; bump a section back toward the left margin
\newenvironment{unindentsection}[1]%
{\begin{list}{}%
	{\setlength{\leftmargin}{-0.5#1}}%
	\item[]%
}
{\end{list}}

% format two pieces of text, one left aligned and one right aligned
\newcommand{\headerrow}[2]
{\begin{tabular*}{\linewidth}{l@{\extracolsep{\fill}}r}
	#1 &
	#2 \\
\end{tabular*}}

% make "C++" look pretty when used in text by touching up the plus signs
\newcommand{\CPP}
{C\nolinebreak[4]\hspace{-.05em}\raisebox{.22ex}{\footnotesize\bf ++}}

% and the actual content starts here
\begin{document}

\begin{center}
{\LARGE \textbf{Mia Feng}}

\url{https://skaudrey.github.io} \textbullet
\ \ \texttt{fengmiao16@nudt.edu.cn} \textbullet
\ \ +86 132 7223 6003 
\\
National University of Defense Technology, Wuhan University\ \ \textbullet
\ \ Changsha city, Hunan province, China
\end{center}


\hrule
\vspace{-0.4em}
\subsection*{Education}

\begin{itemize}
	\parskip=0.1em
	
	\item 
	\headerrow
		{\textbf{National University of Defense Techonology}}
		{\textbf{Changsha city, China}}
	\\
	\headerrow
		{\emph{School of Computing, M.E. Computer science and techonology}}
		{\emph{9/2016 -- Present}}
	\begin{itemize*}
		\item My main research interests are data analysis, reinforce learning, and transfer learning.
		\item relevant courses: CS229, CS231n, Linear Algebra\href{https://ocw.mit.edu/courses/mathematics/18-06-linear-algebra-spring-2010/video-lectures/}.
		\item relevant reading: GP for Machine Learning\href{http://www.gaussianprocess.org/gpml/}, Psychology\href{https://scholar.harvard.edu/schacterlab/pages/publications}.
	\end{itemize*}
	
	\item 
	\headerrow
		{\textbf{Wuhan University}}
		{\textbf{Wuhan city, China}}
	\\
	\headerrow
		{\emph{International software school, B.E. Spatial Informatics and Digitalized Technology}}
		{\emph{09/2012 -- 06/2016}}
	\begin{itemize*}
		\item National Scholarship, Aug.,2015; Excellent Student Cadre, Aug.,2015.
		\item Final GPA: 3.5; thesis: \emph{Fast Satellite Image Storage and Plugin
Development Based On HDFS}
		\item relevant courses: Statistics, Fundemental of Physics, Advanced mathematics,  Linear Algebra.
	\end{itemize*}

	\item 
	\headerrow
		{\textbf{Wuhan University}}
		{\textbf{Wuhan city, China}}
	\\
	\headerrow
		{\emph{Economics and management school, B.S. Finance}}
		{\emph{09/2013 -- 06/2016}}
	\begin{itemize*}
		\item Final grade: 3.0; thesis: \emph{Implications of the Financial Crisis
Inherent Defects of the International Monetary System and Some Advice}
	\end{itemize*}
	
\end{itemize}

\hrule
\vspace{-0.4em}
\subsection*{Experience}

\begin{itemize}
	\parskip=0.1em
	
	\item
	\headerrow
		{\textbf{\href{https://about.meituan.com/}{Meituan-Dianping}}}
		{\textbf{Beijing, China}}
	\\
	\headerrow
		{\emph{Research and Development engineer, fintech}}
		{\emph{07/2018 -- 09/2018}}
	\begin{itemize*}
		\item Worked on anti-fraud detection
		\item I proposed three patents relate to anti-fraud detection, identification detection and intention detection. The patents have been accepted by Meituan-Dianping, and will be handed by them.\footnote{\url{https://developer.aylien.com/text-api-demo?tab=sentiment}}.
	\end{itemize*}

%	\item
%	\headerrow
%		{\textbf{IBM}}
%		{\textbf{Munich, Germany}}
%	\\
%	\headerrow
%		{\emph{Extreme Blue Intern, Watson}}
%		{\emph{08/2015 -- 09/2015}}
%	\begin{itemize*}	
%		\item Design and implementation of text analysis ML components applied to customer data of leading German insurance company \emph{Versicherungskammer Bayern}; automatically identifies structural semantics and sentiment of incoming e-mails, e.g. complaints and classifies email based on reason for complaint.
%		\item Pitched project to audience at European Expo and was chosen as one of eight teams to pitch to IBM customers; project was referred to as a "lighthouse project for Watson in Europe" by jury members.
%		\item Project was awarded Digital Thought Leadership award in leading contest of German insurance industry by leading German newspaper \emph{Süddeutsche Zeitung} and Google\footnote{\url{https://www.sv-veranstaltungen.de/site/fachbereiche/versicherungs-leuchtturm}} and covered by \emph{Süddeutsche Zeitung}\footnote{\url{http://www.sueddeutsche.de/wirtschaft/kuenstliche-intelligenz-aerger-fuer-watson-1.2772927}}.
%	\end{itemize*}
%
%	\item
%	\headerrow
%		{\textbf{Microsoft}}
%		{\textbf{Dublin, Ireland}}
%	\\
%	\headerrow
%		{\emph{Linguistic Engineering Intern, XBox}}
%		{\emph{02/2015 -- 06/2015}}
%	\begin{itemize*}
%		\item Contributed to developing an ML system for analyzing linguistic complexity of strings in C\# for localization prioritization during testing; performed feature analysis and framed problem as anomaly detection.
%		\item Created proof of concept and implemented morphology-based terminology validation algorithm.
%		\item Evangelized customer sentiment analysis efforts, drove cross-team collaboration, and provided insights to stakeholders.
%	\end{itemize*}
%
%	\item
%	\headerrow
%		{\textbf{The OpenCog Foundation}}
%		{\url{opencog.org}}
%	\\
%	\headerrow
%		{\emph{Google Summer of Code Intern}}
%		{\emph{Summer 2014}}
%	\begin{itemize*}
%		\item Implemented deductive reasoning algorithms to enable a model to make common-sense inferences, e.g. \emph{All men are mortal. Socrates is a man.} $\rightarrow$ \emph{Socrates is mortal.}
%		\item Applied inference using probabilistic logic networks on the output of a relationship extractor.
%		\item Documented and extended Python code for temporal inference.
%	\end{itemize*}
%
%	\item
%	\headerrow
%		{\textbf{Lingenio GmbH}}
%		{\textbf{Heidelberg, Germany}}
%	\\
%	\headerrow
%		{\emph{Software Engineering Intern}}
%		{\emph{Spring 2014}}
%	\begin{itemize*}
%		\item Created a converter from TBX to Lingenio native format and vice versa.
%		\item Integrated TBX term bases in Dictionary Server; created localized web service using Jinja2, Flask-Babel, and lighttpd.
%	\end{itemize*}
%
%	\item
%	\headerrow
%		{\textbf{SAP}}
%		{\textbf{Walldorf, Germany}}
%	\\
%	\headerrow
%		{\emph{Working Student, Development University}}
%		{\emph{02/2013 -- 02/2014}}
%	\begin{itemize*}
%		\item Created content for internal programming and Design Thinking courses.
%		\item Automated reporting processes, e.g. reduced expenditure of work for monthly training report from 8 hours to 2 hours using Excel / VBA scripts.
%	\end{itemize*}
%
%	\item
%	\headerrow
%		{\textbf{TEMIS}}
%		{\textbf{Heidelberg, Germany}}
%	\\
%	\headerrow
%		{\emph{Freelancing Developer}}
%		{\emph{02/2013 -- 10/2013}}
%	\begin{itemize*}
%		\item Created a cosine metric-based word sense disambiguation system leveraging text extracted from Wikipedia and DBpedia dumps; achieved performance comparable to the state-of-the-art.
%	\end{itemize*}

\end{itemize}


\hrule
\vspace{-0.4em}
\subsection*{Certificates and awards}

\begin{itemize}
	\parskip=0.1em
	
	\item 
	\headerrow
		{Google Developer Expert -- Machine Learning}
		{\emph{12/2017 -- Present}}
	\item 
	\headerrow
		{Scholarship of the Irish Research Council}
		{\emph{10/2015 -- Present}}
	\item 
	\headerrow
		{Scholarship of the \emph{Cusanuswerk}, one of the 13 German sponsorship organizations}
		{\emph{04/2014 -- 09/2015}}	
	\item 
	\headerrow
		{Microsoft Certified Professional (Programming in C\#)}
		{\emph{06/2015}}
	\item 
	\headerrow
		{Best Delegate award in various Model United Nations conferences}
		{\emph{11/2012 -- 01/2014}}
	\item 
	\headerrow
		{Second and third prizes \emph{Bundeswettbewerb Fremdsprachen}, national foreign languages competition}
		{\emph{2007 -- 2008}}
	\item 
	\headerrow
		{First and second prizes \emph{Landeswettbewerb Mathematik}, state mathematics competition}
		{\emph{2006 -- 2008}}
	

\end{itemize}

\hrule
\vspace{-0.4em}
\subsection*{Languages and Technologies}

\begin{indentsection}{\parindent}
\hyphenpenalty=1000
\begin{description*}
	\item[Programming Languages:]
	Python, Java, C++, \LaTeX, Matlab, JavaScript, SQL
	\item[Technologies:]
	SciPy, NumPy, Keras, TensorFlow, DyNet, scikit-learn, UNIX, Git
	\item[Natural Languages:]
	Fluent in Chinese and English, beginner in French and Japanese.
%	\item[Open Source Contributions:]
%	The OpenCog Foundation
\end{description*}
\end{indentsection}

%\hrule
%\vspace{-0.4em}
%\subsection*{Other activities}
%
%\begin{itemize}
%	\parskip=0.1em
%		\item 
%	\headerrow
%		{\textbf{Natural Language Processing Dublin organizer}}
%		{\emph{08/2016 -- Present}}
%	\begin{itemize*}
%		\item Organized 10 events. Meetup\footnote{\url{https://www.meetup.com/NLP-Dublin/}} has 600+ members and connects students, researchers, and industry professionals.
%	\end{itemize*}
%
%\end{itemize}

\hrule
\vspace{-0.4em}
\subsection*{Publications}

\begin{enumerate}
	\parskip=0.1em
	
	\item \textbf{Feng M}, Zhang W, Zhu X, et al. Multivariate Interpolation of Wind Field Based on Gaussian Process Regression[J]. Atmosphere, 2018, 9(5):194.
	
%	\newenvironment{starfootnotes}
%  {\par\edef\savedfootnotenumber{\number\value{footnote}}
%   \renewcommand{\thefootnote}{$\star$} 
%   \setcounter{footnote}{0}}
%  {\par\setcounter{footnote}{\savedfootnotenumber}}
%	
%\begin{starfootnotes}
%\footnotetext{Equal contribution.}
%\end{starfootnotes}

\end{enumerate}

%\hrule
%\vspace{-0.4em}
%\subsection*{Services to the community}
%
%\begin{itemize}
%	\parskip=0.1em
%	
%	\item \textbf{Reviewer for journals:} Transactions on Audio, Speech and Language Processing; Artificial Intelligence; IEEE Computational Intelligence Magazine
%	
%	\item \textbf{Reviewer for workshops:} RELNLP 2018, DeepLo 2018, SemEval-2016 Task 5
%		
%	\item \textbf{Reviewer for conferences:} ICML 2019, NAACL-HLT 2019, ICLR 2019, ACL 2018, EMNLP 2018, CoNLL 2018
%	
%	\item \textbf{Organizer}: NLP Dublin Meetup\footnote{\url{https://www.meetup.com/NLP-Dublin/}}, NLP Session at Deep Learning Indaba 2018\footnote{\url{http://www.deeplearningindaba.com/schedule-2018.html}}
%
%\end{itemize}

\hrule
\vspace{-0.4em}
\subsection*{Talks}

\begin{itemize}
	\parskip=0.1em
	
	\item Discussion about Data Assimilation and Machine Learning, Sep. 11th, 2017.\footnote{\url{https://skaudrey.github.io/assets/slides/D.A/pres.pdf}}

	\item Multivariate Interpolation of Wind Fields Based on Gaussian Process Regression, Jan. 24th, 2018.\footnote{\url{https://skaudrey.github.io/assets/slides/gpr/windInterpolation.pdf}}
	
	\item The Introduction of Infrared Hyper-spectrtum Data and kernel PCA compression, June 5th, 2018.\footnote{\url{https://skaudrey.github.io/assets/slides/hyp/hypCompression.pdf}}
	
\end{itemize}

\end{document}
