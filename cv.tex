% resume.tex
% vim:set ft=tex spell:

\documentclass[10pt,letterpaper]{article}
\usepackage[letterpaper,margin=0.75in]{geometry}
\usepackage[utf8]{inputenc}
\usepackage{mdwlist}
\usepackage[T1]{fontenc}
\usepackage{textcomp}
\usepackage{tgpagella}
\usepackage{latexsym}
\usepackage{amssymb}
\usepackage{hyperref}
\pagestyle{empty}
\setlength{\tabcolsep}{0em}

% indentsection style, used for sections that aren't already in lists
% that need indentation to the level of all text in the document
\newenvironment{indentsection}[1]%
{\begin{list}{}%
	{\setlength{\leftmargin}{#1}}%
	\item[]%
}
{\end{list}}

% opposite of above; bump a section back toward the left margin
\newenvironment{unindentsection}[1]%
{\begin{list}{}%
	{\setlength{\leftmargin}{-0.5#1}}%
	\item[]%
}
{\end{list}}

% format two pieces of text, one left aligned and one right aligned
\newcommand{\headerrow}[2]
{\begin{tabular*}{\linewidth}{l@{\extracolsep{\fill}}r}
	#1 &
	#2 \\
\end{tabular*}}

% make "C++" look pretty when used in text by touching up the plus signs
\newcommand{\CPP}
{C\nolinebreak[4]\hspace{-.05em}\raisebox{.22ex}{\footnotesize\bf ++}}

% and the actual content starts here
\begin{document}

\begin{center}
{\LARGE \textbf{Mia Miao Feng}}

\url{https://skaudrey.github.io} \textbullet
\ \ \texttt{fengmiao16@nudt.edu.cn} \textbullet
\ \ +86 132 7223 6003 
\\
National University of Defense Technology, Wuhan University\ \ \textbullet
\ \ Changsha city, Hunan province, China
\end{center}


\hrule
\vspace{-0.4em}
\subsection*{Education}

\begin{itemize}
	\parskip=0.1em
	
	\item 
	\headerrow
		{\textbf{National University of Defense Techonology}}
		{\textbf{Changsha city, China}}
	\\
	\headerrow
		{\emph{School of Computing, M.E., Computer science and techonology}}
		{\emph{9/2016 -- Present}}
	\begin{itemize*}
		\item My main research interests are data analysis, reinforce learning and transfer learning and visualization and explanation of neural networks.
               \item Thesis: The Study of Typical Applications in Weather Forecasting Based on Machine Learning.
		\item relevant courses: CS229, CS231n, Linear Algebra\href{https://ocw.mit.edu/courses/mathematics/18-06-linear-algebra-spring-2010/video-lectures/}.
		\item relevant reading: Gaussian Process for Machine Learning\href{http://www.gaussianprocess.org/gpml/}, Deep Learning, and Psychology\href{https://scholar.harvard.edu/schacterlab/pages/publications}.
	\end{itemize*}
	
	\item 
	\headerrow
		{\textbf{Wuhan University}}
		{\textbf{Wuhan city, China}}
	\\
	\headerrow
		{\emph{International software school, B.E.\href{https://github.com/skaudrey/cv/blob/master/certificate/certificate.pdf}, Spatial Informatics and Digitalized Technology}}
		{\emph{09/2012 -- 06/2016}}
	\begin{itemize*}
		\item National Scholarship, Aug.,2015; Excellent Student Cadre, Aug.,2015.
		\item Final GPA\href{https://github.com/skaudrey/cv/blob/master/certificate/major transcript.pdf}: 3.5\href{https://github.com/skaudrey/cv/blob/master/certificate/major gpa.pdf}; thesis: \emph{Fast Satellite Image Storage and Plugin
Development Based on HDFS.}
		\item relevant courses: Statistics, Fundemental of Physics, Advanced mathematics,  Linear Algebra.
	\end{itemize*}

	\item 
	\headerrow
		{\textbf{Wuhan University}}
		{\textbf{Wuhan city, China}}
	\\
	\headerrow
		{\emph{Economics and management school, B.S.\href{https://github.com/skaudrey/cv/blob/master/certificate/certificate.pdf}, Finance.}}
		{\emph{09/2013 -- 06/2016}}
	\begin{itemize*}
		\item Final GPA\href{https://github.com/skaudrey/cv/blob/master/certificate/minor transcript.pdf}: 3.0\href{https://github.com/skaudrey/cv/blob/master/certificate/minor gpa.pdf}; thesis: \emph{Implications of the Financial Crisis
Inherent Defects from International Monetary System and Some Advice.}
	\end{itemize*}
	
\end{itemize}

\hrule
\vspace{-0.4em}
\subsection*{Experience}

\begin{itemize}
	\parskip=0.1em
	
	\item
	\headerrow
		{\textbf{\href{https://about.meituan.com/}{Meituan-Dianping}}}
		{\textbf{Beijing, China}}
	\\
	\headerrow
		{\emph{Research and Development engineer, fintech}}
		{\emph{07/2018 -- 09/2018}}
	\begin{itemize*}
		\item Worked on anti-fraud detection
		\item I proposed three patents related to anti-fraud detection, identification detection and intention detection. The patents have been accepted by Meituan-Dianping, and will be handed by them.\footnote{\url{https://github.com/skaudrey/cv/blob/master/patent/list.png}}.
	\end{itemize*}

\end{itemize}


\hrule
\vspace{-0.4em}
\subsection*{Certificates and awards}

\begin{itemize}
	\parskip=0.1em
	
	\item 
	\headerrow
		{Outstanding Organizer}
		{\emph{12/2016}}
	\item 
	\headerrow
		{3rd prize, \href{http://gmcm.seu.edu.cn/}The 13th MCM of Master}
		{\emph{90/2016}}
	\item 
	\headerrow
		{Excellent Graduate}
		{\emph{06/2016}}
	\item 
	\headerrow
		{National Scholarship}
		{\emph{08/2015}}	
	\item 
	\headerrow
		{Outstanding Student Leader}
		{\emph{08/2015}}
	\item 
	\headerrow
		{2nd Prize, COMAP's \href{https://www.comap.com/undergraduate/contests/}MCM}
		{\emph{02/2015}}
\end{itemize}

\hrule
\vspace{-0.4em}
\subsection*{Languages and Technologies}

\begin{indentsection}{\parindent}
\hyphenpenalty=1000
\begin{description*}
	\item[Programming Languages:]
	Python, Java, C++, \LaTeX, Matlab, JavaScript, SQL
	\item[Technologies:]
	SciPy, NumPy, Keras, TensorFlow, DyNet, scikit-learn, UNIX, Git
	\item[Natural Languages:]
	Fluent in Chinese and English, beginner in French and Japanese.
%	\item[Open Source Contributions:]
%	The OpenCog Foundation
\end{description*}
\end{indentsection}

\hrule
\vspace{-0.4em}
\subsection*{Projects}

\begin{itemize}
	\parskip=0.1em
		\item 
	\headerrow
		{\textbf{Naive implementations of some popular machine learning algorithms.}}
		{\emph{03/2018 -- Present}}
	\begin{itemize*}
		\item Naive implementations of some M.L. algorithms, which are updated continuously\footnote{\url{https://skaudrey.github.io/posts/projects/2018-11-16-ml-implement.html}}.
	\end{itemize*}

\parskip=0.1em
		\item 
	\headerrow
		{\textbf{HCR--Compress and Resonstruct hyperspectral data.}}
		{\emph{10/2018}}
	\begin{itemize*}
		\item  A network for compressing and reconstructing infrared hyperspectral data, named HCR, is proposed\footnote{\url{https://skaudrey.github.io/posts/projects/2018-11-16-hcr.html}}.
	\end{itemize*}

\parskip=0.1em
		\item 
	\headerrow
		{\textbf{Cloud detection of infrared hyperspectral data based on logistic.}}
		{\emph{04/2018}}
	\begin{itemize*}
		\item  Detect whether infrared atmospheric sounding interferometer’s (IASI’s) instantaneous fields of view (IFOVs) are covered by clouds. Based on the proposed feature construction method, cloudy IFOVs are detected by logistic regression, which aims at forecasting in real time\footnote{\url{https://skaudrey.github.io/posts/projects/2018-11-16-lr.html}}.
	\end{itemize*}

\parskip=0.1em
		\item 
	\headerrow
		{\textbf{Interpolating weather processes based on Gaussian Process Regression.}}
		{\emph{06/2017--08/2017}}
	\begin{itemize*}
		\item  Interpolating wind fields. Design a multi-scale anisotropy kernel for weather processes, and two multivariate models for interpolating weather processes with and without cyclones are proposed\footnote{\url{https://skaudrey.github.io/posts/projects/2018-11-11-gpr.html}}.
	\end{itemize*}


\end{itemize}

\hrule
\vspace{-0.4em}
\subsection*{Publications}

\begin{enumerate}
	\parskip=0.1em
	
	\item \textbf{Feng M}, Zhang W, Zhu X, et al. Multivariate Interpolation of Wind Field Based on Gaussian Process Regression[J]. Atmosphere, 2018, 9(5):194.
	
%	\newenvironment{starfootnotes}
%  {\par\edef\savedfootnotenumber{\number\value{footnote}}
%   \renewcommand{\thefootnote}{$\star$} 
%   \setcounter{footnote}{0}}
%  {\par\setcounter{footnote}{\savedfootnotenumber}}
%	
%\begin{starfootnotes}
%\footnotetext{Equal contribution.}
%\end{starfootnotes}

\end{enumerate}

\hrule
\vspace{-0.4em}
\subsection*{Talks}

\begin{itemize}
	\parskip=0.1em
	
	\item Discussion about Data Assimilation and Machine Learning, Sep. 11th, 2017.\footnote{\url{https://skaudrey.github.io/posts/talks/2018-11-12-da+talk.html}}

	\item Multivariate Interpolation of Wind Fields Based on Gaussian Process Regression, Jan. 24th, 2018.\footnote{\url{https://skaudrey.github.io/posts/talks/2018-11-16-gpr-talk.html}}
	
	\item The Introduction of Infrared Hyperspectral Data and Kernel PCA, June 5th, 2018.\footnote{\url{https://skaudrey.github.io/posts/talks/2018-11-12-hyp+talk.html}}

	\item What Can Artificial Intelligence Do in Data Assimilation? December 9th, 2018.\footnote{\url{https://skaudrey.github.io/posts/talks/2018-12-10-mlutility+talk.html}}
	
\end{itemize}

\hrule
\vspace{-0.4em}
\subsection*{Other acdemic activities}

\begin{itemize}
	\parskip=0.1em
	
	\item \textbf{The International Summer School on Applied Mathematics:} Machine Learning, Deep Learning, Data Assimilation, Statistical Inference in high dimensions.\footnote{\url{https://skaudrey.github.io/posts/meetings/2018-11-13-harbin.html}}.
	
	\item \textbf{Computing in the 21th Century \& Asia Faculty Summit:} Microsoft, Computer Science, AI, Computational biology.\footnote{\url{https://skaudrey.github.io/posts/meetings/2018-11-13-microsoft.html}}.

\end{itemize}

\end{document}
