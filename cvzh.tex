% resume.tex
% vim:set ft=tex spell:

\documentclass[10pt,letterpaper]{article}
\usepackage[letterpaper,margin=0.75in]{geometry}
\usepackage[utf8]{inputenc}
\usepackage{mdwlist}
\usepackage[T1]{fontenc}
\usepackage{textcomp}
\usepackage{tgpagella}
\usepackage{latexsym}
\usepackage{amssymb}
\usepackage{hyperref}
% Chinese
\usepackage{xeCJK}

\pagestyle{empty}
\setlength{\tabcolsep}{0em}

% indentsection style, used for sections that aren't already in lists
% that need indentation to the level of all text in the document
\newenvironment{indentsection}[1]%
{\begin{list}{}%
	{\setlength{\leftmargin}{#1}}%
	\item[]%
}
{\end{list}}

% opposite of above; bump a section back toward the left margin
\newenvironment{unindentsection}[1]%
{\begin{list}{}%
	{\setlength{\leftmargin}{-0.5#1}}%
	\item[]%
}
{\end{list}}

% format two pieces of text, one left aligned and one right aligned
\newcommand{\headerrow}[2]
{\begin{tabular*}{\linewidth}{l@{\extracolsep{\fill}}r}
	#1 &
	#2 \\
\end{tabular*}}

% make "C++" look pretty when used in text by touching up the plus signs
\newcommand{\CPP}
{C\nolinebreak[4]\hspace{-.05em}\raisebox{.22ex}{\footnotesize\bf ++}}

% and the actual content starts here
\begin{document}

\begin{center}
{\LARGE \textbf{冯淼}}

Blog:  \url{https://skaudrey.github.io} \textbullet
\ \ \texttt{Email: fengmiao16@nudt.edu.cn} \textbullet
\ \ +86 132 7223 6003 
\\
国防科技大学, 武汉大学\ \ \textbullet
\ \ 湖南长沙,中国
\end{center}


\hrule
\vspace{-0.4em}
\subsection*{教育}

\begin{itemize}
	\parskip=0.1em
	
	\item 
	\headerrow
		{\textbf{国防科技大学}}
		{\textbf{中国长沙}}
	\\
	\headerrow
		{\emph{计算机科学与技术学院,工学硕士,计算机科学与技术专业}}
		{\emph{9/2016 -- 至今}}
	\begin{itemize*}
		\item 主要研究兴趣为数据分析,图深度学习,机器学习基础。
		\item GPA: 3.5/4.0.
               \item 毕设:机器学习在天气预报的典型应用研究
		\item 相关课程: CS229, CS231n, 线代\href{https://ocw.mit.edu/courses/mathematics/18-06-linear-algebra-spring-2010/video-lectures/}.
		\item 相关阅读: {\emph{Gaussian Process for Machine Learning}}\href{http://www.gaussianprocess.org/gpml/},  {\emph{Deep Learning}}, {\emph{PRML}}.
%, and Psychology\href{https://scholar.harvard.edu/schacterlab/pages/publications}
	\end{itemize*}
	
	\item 
	\headerrow
		{\textbf{武汉大学}}
		{\textbf{中国武汉}}
	\\
	\headerrow
		{\emph{国际软件学院, 工学学士\href{https://github.com/skaudrey/cv/blob/master/certificate/certificate.pdf}, 空间信息与数字技术(主修)}}
		{\emph{09/2012 -- 06/2016}}
	\begin{itemize*}
		\item GPA\href{https://github.com/skaudrey/cv/blob/master/certificate/major transcript.pdf}: 3.6/4.0\href{https://github.com/skaudrey/cv/blob/master/certificate/major gpa.pdf}.
		\item 毕设: \emph{基于HDFS的卫星影像快速存储与插件开发}
		\item 相关课程: 统计、物理基础、高数、线代。
	\end{itemize*}

	\item 
	\headerrow
		{\textbf{武汉大学}}
		{\textbf{中国武汉}}
	\\
	\headerrow
		{\emph{经济与管理学院, 理学学士\href{https://github.com/skaudrey/cv/blob/master/certificate/certificate.pdf}, 金融学(辅修)}}
		{\emph{09/2013 -- 06/2016}}
	\begin{itemize*}
		\item GPA\href{https://github.com/skaudrey/cv/blob/master/certificate/minor transcript.pdf}: 3.1/4.0\href{https://github.com/skaudrey/cv/blob/master/certificate/minor gpa.pdf}.
		\item 毕设:{ \emph{金融危机的启示——
国际货币体系的内在缺陷及改进方法}}
	\end{itemize*}
	
\end{itemize}

\hrule
\vspace{-0.4em}
\subsection*{语言与技能}

\begin{indentsection}{\parindent}
\hyphenpenalty=1000
\begin{description*}
	\item[编程语言:]
	Python, Java, C++, \LaTeX, Matlab, JavaScript, SQL
	\item[技能:]
	SciPy, NumPy, Keras, TensorFlow, scikit-learn, UNIX, Git
	\item[外语:]
	精通汉语、英语, 法语、日语入门。
%	\item[Open Source Contributions:]
%	The OpenCog Foundation
\end{description*}
\end{indentsection}




\hrule
\vspace{-0.4em}
\subsection*{项目}

\begin{itemize}
	\parskip=0.1em
		\item 
	\headerrow
		{\textbf{常见机器学习算法的简单实现}}
		{\emph{03/2018 -- 现在}}
	\begin{itemize*}
		\item 一些常见机器学习算法的原型实现,持续更新中。\footnote{\url{https://skaudrey.github.io/posts/projects/2018-11-16-ml-implement.html}}
	\end{itemize*}

\parskip=0.1em
		\item 
	\headerrow
		{\textbf{HCR--压缩、重构红外高光谱数据}}
		{\emph{10/2018}}
	\begin{itemize*}
		\item  CNN网络,取名为HCR,用于压缩与重构红外高光谱数据。\footnote{\url{https://skaudrey.github.io/posts/projects/2018-11-16-hcr.html}}
	\end{itemize*}

\parskip=0.1em
		\item 
	\headerrow
		{\textbf{基于logistic的红外高光谱资料云检测}}
		{\emph{04/2018}}
	\begin{itemize*}
		\item  检测IASI(红外大气探测干涉仪)的IFOVs(瞬时视场)是否被云覆盖。基于提出的特征构建方法,采用logistic算法对视场进行实时检测。\footnote{\url{https://skaudrey.github.io/posts/projects/2018-11-16-lr.html}}
	\end{itemize*}

\parskip=0.1em
		\item 
	\headerrow
		{\textbf{基于高斯过程回归的天气过程插值}}
		{\emph{06/2017--08/2017}}
	\begin{itemize*}
		\item 插值风场。为天气过程设计了多尺度各向异性核函数,并且针对有气旋天气过程和无气旋的天气过程分别设计了多变量插值模型\footnote{\url{https://skaudrey.github.io/posts/projects/2018-11-11-gpr.html}}。
	\end{itemize*}


\end{itemize}

\hrule
\vspace{-0.4em}
\subsection*{论文}

\begin{enumerate}
	\parskip=0.1em
	
	\item \textbf{Feng M}, Zhang W, Zhu X, et al. Multivariate Interpolation of Wind Field Based on Gaussian Process Regression[J]. Atmosphere, 2018, 9(5):194.
	
%	\newenvironment{starfootnotes}
%  {\par\edef\savedfootnotenumber{\number\value{footnote}}
%   \renewcommand{\thefootnote}{$\star$} 
%   \setcounter{footnote}{0}}
%  {\par\setcounter{footnote}{\savedfootnotenumber}}
%	
%\begin{starfootnotes}
%\footnotetext{Equal contribution.}
%\end{starfootnotes}

\end{enumerate}

\hrule
\vspace{-0.4em}
\subsection*{报告}

\begin{itemize}
	\parskip=0.1em
	
	\item 资料同化与机器学习关系的讨论,09/11/2017。\footnote{\url{https://skaudrey.github.io/posts/talks/2018-11-12-da+talk.html}}

	\item 基于高斯过程的风场多变量插值,01/24/2018。\footnote{\url{https://skaudrey.github.io/posts/talks/2018-11-16-gpr-talk.html}}
	
	\item 红外高光谱数据与KPCA介绍,01/05/2018。\footnote{\url{https://skaudrey.github.io/posts/talks/2018-11-12-hyp+talk.html}}

	\item 人工智能在资料同化中可以做什么?12/09/2018。\footnote{\url{https://skaudrey.github.io/posts/talks/2018-12-10-mlutility+talk.html}}
	
\end{itemize}

\hrule
\vspace{-0.4em}
\subsection*{学术活动}

\begin{itemize}
	\parskip=0.1em
	
	\item \textbf{国际应用数学暑期学校:} 机器学习,深度学习,资料同化,高维统计推断。\footnote{\url{https://skaudrey.github.io/posts/meetings/2018-11-13-harbin.html}}
	
	\item \textbf{\"二十一世纪的计算\"学术研讨会:} 微软,计算机科学,AI,计算生物学。\footnote{\url{https://skaudrey.github.io/posts/meetings/2018-11-13-microsoft.html}}

\end{itemize}

\hrule
\vspace{-0.4em}
\subsection*{实习经历}

\begin{itemize}
	\parskip=0.1em
	
	\item
	\headerrow
		{\textbf{\href{https://about.meituan.com/}{美团点评}}}
		{\textbf{北京,中国}}
	\\
	\headerrow
		{\emph{算法研发,金融反欺诈}}
		{\emph{07/2018 -- 09/2018}}
	\begin{itemize*}
		\item 负责利用机器学习算法进行金融反欺诈行为检测。
		\item 在岗期间提出了三项专利,分别涉及发欺诈检测、身份识别与意图鉴定,均已被美团点评公司录用。后续会成文递交至专利局。\footnote{\url{https://github.com/skaudrey/cv/blob/master/patent/list.png}}
	\end{itemize*}

\end{itemize}


\hrule
\vspace{-0.4em}
\subsection*{奖励}

\begin{itemize}
	\parskip=0.1em
	
	\item 
	\headerrow
		{三等奖,\href{http://gmcm.seu.edu.cn/}第十三届全国研究生数学建模竞赛}
		{\emph{09/2016}}
	\item 
	\headerrow
		{优秀毕业生}
		{\emph{06/2016}}
	\item 
	\headerrow
		{国家奖学金}
		{\emph{08/2015}}	
	\item 
	\headerrow
		{校级优秀学生干部}
		{\emph{08/2015}}
	\item 
	\headerrow
		{二等奖,美国大学生数学建模竞赛\href{https://www.comap.com/undergraduate/contests/}MCM}
		{\emph{02/2015}}
\end{itemize}

\end{document}
